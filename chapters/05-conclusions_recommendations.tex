\chapter{Conclusions and Recommendations}

For high complexity copter vehicles like the OMAV at ASL, it is imperative to have a good understanding of how actuators react to commands, to maintain stability or to be able to design more aggressive controllers with less robustness to model inaccuracies. Therefore, this project aimed to investigate sensing and performance characteristics of modern ESCs relevant to control objectives. Especial focus is given to the factors that affect accurate throttle speed to rotor mapping.\\

During the project, a thorough investigation of the working principles governing BLDC motors and ESCs was performed. For instance, the different methods of driving BLDC motors are introduced concluding that at high speeds there is little difference between FOC and trapezoidal based ESCs. Different FC to ESC communication protocols were also discussed concluding that digital protocols are key to obtain ESC feedback. Current research has proven that throttle to rotor speed mapping is dependent on voltage. However, most of their models try to simplify the system to a linear  model. A significant amount of research has been focused on efficiency of ESCs.\\

The "TEKKO32 F3 Metal 4 in 1" ESC with BlHeli\_32 capabilities was used and examined. To do so,  a software tool to utilize the Dshot protocol under the PX4 framework with a Pixhawk 4 was developed. It allowed to run motors via Nutshell and receive rotor status information including rotor speed, current consumed and battery voltage. This feedback signal was obtained at $300Hz$ for a single motor and $75Hz$ for 4 motors simultaneously. The hardware was setup in a rig mimicking coaxial rotor pairs in the OMAV setup.\\

Experimental data suggests that the current and rotor speed measurements given by the ESC are very accurate provided the current ADC is calibrated. Furthermore, a model quadratic on throttle and linear on voltage used to describe rotor speed showed to be representative with errors below $1.5\%$. Different ESCs of the same model response were analyzed for different motors, showing that there is significant variability of throttle to RPM mapping caused by ESC and motor inconsistencies.
Besides, the active brake in the ESC is always active and allows for time constants of $52.5ms$ on falling steps and $49ms-80ms$ on rising edges. Coaxial propellers cause significant interference that increases the lower propeller size up to  $5.2\%$. Lastly, for BlHeli\_32 based ESCs, higher PWM frequencies in the motor driving signals showed higher efficiency up to $48kHz$ which is the maximum modulation frequency allowed in this hardware. However, for much higher frequencies, efficiency is expected to lower.\\

\textit{Recommendations. } \\
The main recommendations to be presented are dependent on the use or not of a Pixhawk 4. This FC limits the number of of Dshot outputs to 4 (or 8 with significant change to PX4). PX4 does not support Bidirectional Dshot either.\\

Therefore, if not using a Pixhawk 4, and ESC with Bidirectional Dshot capabilities (such as the one used in this project) is recommended as it allows telemetry feedback through the same lines that throttle outputs are run allowing for real-time rotor speed feedback. Then, better rotor control can be performed in the FC.\\

If using a PX4, the best option would be to use a UAVCAN based ESC setup to the maximum baud-rate allowed of $1Mbit/s$, allowing to obtain the same data as in Dshot protocol but at higher speeds. A good suggestion of this kind of ESCs is the Zubax Myxa motor, that support UAVCAN and rotor speed control internally \cite{Zubax-Robotics2015}.\\

Lastly, if only feed-forward control is desired, proper calibration for each motor-esc pair need to be performed in order to achieve desired rotor speeds when commanding throttle values. This calibration also needs to account for variations in voltage. I.e. the feed-forward model fitted needs to be a 3 dimensional, ideally non-linear on throttle and linear on voltage.